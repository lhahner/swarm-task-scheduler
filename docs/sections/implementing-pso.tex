\section{Implementing the particle swarm optimization algorithm in Java}
\paragraph{Project enviroment}
The project was setup with the project management tool maven and the spring initalizer.
This enables a portable solution and solves every dependency quickly.
Furthermore was the algorithm implemented with the Intergrated Development Enviroment (IDE)
Eclipse. 

\paragraph{Object-oriented concept}
Before actually implementing the algorithm itself, 
the concept was built in an object-oriented manner, in which each particle is based on a 
class. The class Particle has the properties velocity and position, which are implemented 
as variables. Since the velocity is calculated using the presented equation \ref{} and the
 new assigned position \ref{}, both are implemented as methods of the Particle class. 
 Additionally, each particle has an evaluate method that implements the fitness function 
 to optimize.

Each particle implements the interface Agent. This is because the factory design pattern 
is applied for constructing particles in other classes, leading to the implementation of 
the class ParticleFactory. From this class, an object should be constructed, and the 
method getParticle should be called to create particles for the application. The reason 
for this approach is that it ensures that whenever a particle is created, its initial 
position is greater than zero. % Check here again why this should be the case?

Another class, ParticleSwarm, is created to represent the swarm of particles. Here, an 
ArrayList is used to store all the particles that belong to this swarm. Additionally, 
the variable gbest is used to store the global optimum of this swarm.

Together, the classes are constructed as objects inside the ParticleSwarmOptimization 
class. In this class, the evaluation of the fitness function is performed for every 
particle in every iteration of the implemented loop.
To call the optimize method of this class, the user needs to specify the swarm to be 
optimized, the initial values for the particles, and an initial size for the swarm. The 
size of the swarm will determine the number of iterations performed.

% % Listing format
% \begin{figure}[H]
% 	\begin{lstlisting}[language=Java, caption=<description>, 
%         captionpos=b, 
%         label=xl-code,numbers=left ,
%         firstline=1, firstnumber=1, lastline=30]
% 	\end{lstlisting}
% \end{figure}

% % Picture format
% \begin{figure}[H]
% 	\begin{center}
% 		\includegraphics[width=9cm, height=8cm]{<source>}
% 	\end{center}
% 	\caption{<description>}
% 	\label{<label-for-ref>}
% 	\centering
% \end{figure}